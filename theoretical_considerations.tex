
\section{Microcontroller based temperature humidity meter using Arduino Platform}
Arduino is one of the many micro-controller based systems that can be utilized to measure temperature and humidity level. It is a combination of hardware and software computer architecture system that has already made into several versions of small size chipsets. Both of these versions can be used along with the humidity and temperature sensors to detect temperature and humidity in the environment. Temperature and humidity level may vary depends on the locations since every different location are affected by various environments. Different time also affects the result due to the weather change and location of the sun in the sky. 
The Arduino microcontroller system implemented together with the sensor on a device like a portable temperature and humidity meter. The device used and temperature and humidity sensor must have a physical connection and battery for the power supply. The Arduino device will present the data into a LCD display, in order to make it easier for users to read the humidity and temperature levels.

\section{Discomfort Index}
\cite{Discomfort} states that discomfort index refers to impact of heat and stress on the individual taking account the combined effect of temperature and humidity. This index is used as a standard to inform the user whether their respective places are not comfortable or good enough for an activity. Several temperature and humidity levels will be gathered in order to give a more accurate discomfort index as a result. Proper gathering of temperature and humidity level data is necessary to fulfill the purpose of the discomfort index. It is important for student to know which location at the campus is uncomfortable because stress caused by the environments affects the welfare of the students at school.  

\section{Bluetooth Technology}
Bluetooth is a wireless communication technology. This technology deals with the regulation of the flow of data. Data transmission is done though the wireless communication in this technology there are paired two devices and these devices can communicate to each other through Bluetooth. After the paring of devices, there is a process of data transfer. It is a bidirection technology since it is capable of sending and receiving data. It has a limited transmission distance between the two devices and it cannot transmit data in far distances. The temperature and humidity data from the Arduino device can be transferred into the android application in terms of wireless Bluetooth communication as indicated above information.

\section{Comfortability indicator application at De La Salle university using Android platform}
Android is one of the operating system programs that can be used in various purposes. This operating system already has several versions such as Ginger Bread, Ice Cream Sandwich, Jelly Bean, KitKat and Marshmallow. All these versions are compatible with the android operating system to show the comfortability indicator. Marshmallow is the latest version and it has the more functions than the older versions but most phones do not support this version yet. older version of android will be used since it is the version where a lot of students are using it right now. 
The Comfortability Indicator application will based it on the temperature and humidity data gathered in the Arduino device and it will display the heat map to indicate which area is comfortable and which are not. The students will have information about the discomfort index and the effect of heat and stress to their health and welfare.

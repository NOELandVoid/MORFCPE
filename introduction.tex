\section{Background of the Study}

There has been many reasons why one tries to avoid any outdoor activity but one of these is how the air feels whether it is too hot or too polluted or even both. One undeniable fact is that heat and humidity all play roles in making the weather hot. Both of these weather parameters are involved in the calculation of the heat index and the discomfort index.
\par \noindent
Heat index and discomfort index have their similarities because the factors that affect these two are the temperature and the relative humidity. The heat index is the perceived temperature by people when the rising temperature and the relative humidity is combined. The unit used here is a unit of temperature and the mathematical formula for computing the heat index shows a rather direct square proportionality with the temperature and the humidity. But when it comes to a more human readable scale, reaching 34 degrees Celsius is already a discomfort to some. Reaching at least 46 degrees Celsius is already dangerous to all as this can cause heat stroke and even imminent death to some people. The discomfort index is similar to the heat index but instead, its mathematical formula only indicates a direct proportionality with the temperature and the relative humidity. The scaling is rather similar to that in the heat index. When the discomfort index reaches at least 21 degrees Celsius, it is already a discomfort for some people. Reaching 29 degrees Celsius is already dangerous to all that when it even gets higher, a state of emergency can be declared.
\par \noindent
The human body is capable of regulating body temperature because of its abilities as a warm-blooded organism. When the human body detects extreme temperatures, it drastically adjusts the body just to get the internal temperature back to a normal 37 degrees Celsius. When your body detects a lot of heat, it tries to cool itself down by increasing your heart rate and sweating. However, one can sweat too much, he feels drained by the lack of fluids in his body causing discomfort, weakness, loss of stamina, and even muscle pains, leading to a heat stroke.
\par \noindent
Other than high temperatures and humidity, the pollutants in the air can be harmful to the respiratory system. Dust is a particle suspended in the air and it usually comes from the soil or the pollution. This can cause irritation in the respiratory system because dust entering the lungs can cause serious complications. This is already bad for those with respiratory problems such as asthma or emphysema. Carbon monoxide, however, is a colorless and odorless gas and it usually comes from smoke. When this is inhaled, it can cause serious complications in the body since this inhibits the delivery of oxygen from the blood to the other organs in the body which can cause death. Not only do all of this increase the risk of getting sickness but these also affect the visibility of an area.
\par \noindent
This study will focus on a mobile application that enables people to have a foresight on how the outside air feels like. A microcontroller-based system will be used in detecting the parameters stated above while the mobile application will take note of the visibility with the use of the phone's camera.

\section{Prior Studies}

Some of the studies that the group has found are about the temperature and humidity monitoring systems. The temperature system can be constructed by using a simple microcontroller-based system with an important tool, the LM35 where the output voltage is directly proportional to the temperature detected. The same procedure can be done with the humidity sensor but this time, it does not make use of the LM35. Both of these sensors are good for agricultural applications and getting the air quality. Another study involves the use of PM10 sensors in order to detect particulate matter that is 10 micrometers wide. An algorithm has been made with the use of the atmospheric reflectance for temporal monitoring. Another study introduces another concept of air monitoring by taking note of the pollutants present which are namely carbon monoxide, PM 2.5 , and ozone which make use of the MQ-7 4
sensor, MQ-131 sensor, and Sharp dust sensor respectively. Another study made use of getting the discomfort index by using temperature, humidity, atmospheric pressure, and carbon dioxide sensors. Finally, a study states the standards set by different parts of the world when it comes to the air quality. These standards all make use of the amount of pollutants present in the air as basis of air quality.



\section{Problem Statement}

Though there have been mobile applications that display the weather in real time, none have been able to show the discomfort index given the data. Also, there are no applications that tell the amount of dust or carbon monoxide in the air considering that these are some important factors when people choose to commute by an ordinary jeepney or do any outdoor activity in urban areas.
\par \noindent
The aim of this study is to develop a new mobile application that is able to report the condition of the air such as weather parameters and the amount of pollutants present. The system will make use of a microcontroller along with different sensors that will measure the said parameters. Also, the mobile application will make use of computer vision to measure the visibility in an area.
\par \noindent
Can a mobile application be developed to report real time conditions of the air and the amount of pollutants present with the used of a sensor-based microcontroller system?

\section{Objectives}
\subsection{General Objective(s)}
To design and develop an indoor/outdoor system for getting the discomfort index of the air\ldots;

\subsection{Specific Objectives}

\begin{enumerate}

	\item To make use of the temperature, humidity, amount of dust, amount of carbon monoxide, and visibility in calculating discomfort index and measuring pollutants\ldots;

	\item To utilize different sensors for temperature, humidity, dust, and carbon monoxide measurement\ldots;
	
	\item To make use of computer vision with the use of a cellphone camera to measure visibility \ldots;
	
	\item To achieve a social impact on the conditions and quality of the air for the people in urban areas where smoke is present and abundant \ldots;
	
\end{enumerate}



\section{Significance of the Study}

The significance of this topic is to be able to design and produce a device of checking the air quality and discomfort index for the public health awareness. There are millions of commuters in the Philippines riding jeepneys or light rail transit system. The problem of this way of commuting is the air because there are a lot of old vehicles producing smoke and most people just breathe in either direct or indirect way. It is very important for the people to know the status of the air to secure their respiratory health. Together with this, the group aim to the user friendly device that anyone can easily understand how to use the device through an android application. Since a lot of people uses android mobile phones, making an application for free will be very helpful. The application will display the required data in graphics so that it is easy to understand for the public and to make the aware of the effect of the environment to their health. This study will surely help a lot of people who still don’t know about why it is important to know the air we are breathing outside.


\section{Assumptions, Scope and Delimitations}

%Bulletize your scope in one group, and then bulletize the delimitations in another.  Bulletize your assumptions as well.
\begin{enumerate}
\item The given data will only be determined by the air quality index and the discomfort index.

\item The application will be used only for displaying the data gathered in the device.

\item People should be able to know the importance of their respiratory system in the body.

\item Users must aware the connection between air pollution and lung cancer.

\item The device will only deal with the common factors for discomfort such as temperature, humidity, and the amount of dust in the air.

\end{enumerate}

\section{Description and Methodology}

A device for checking air quality and discomfort index can be functional through the use of the electronic sensors attached in the circuit and sensors for dust, humidity, and temperature will provide the data for air quality index and discomfort index. The device will be user friendly so that anyone can easily control and use it for the given purpose.The goal for this project is to come up with a device and android application for air quality and discomfort index which will provide data related to the health of the public.  Challenges to this project would be the design of the circuit with indicated sensors and the accuracy of the data gathered by the device. The size of the device matters because it has to be user friendly and this will be designed for the typical citizens like commuters. The prototype test  would determine if it has accurate data and user friendly in general. Android application will be supporting the device as a method of health awareness. the application will be able to show the data gathered in the device and show the effect of air quality index and discomfort index for respiratory health. The information is also one of the important part because people must know why it is important to know the air quality and their discomfort level.



\section{Estimated Work Schedule and Budget}
% Table generated by Excel2LaTeX from sheet 'Sheet1'
\begin{table}[htbp]
  \centering
  \caption{Gannt Chart Part 1}
    \begin{tabular}{rrrrrrrr}
    \toprule
      &   &       W1 &       W2 &       W3 &       W4 &       W5 &       W6 \\
    \midrule
    Research for a topic &   & X &   &   &   &   &  \\
    Submission of proposed topic &   &   & X &   &   &   &  \\
    Background of the study &   &   &   & X &   &   &  \\
    Statement of the problem &   &   &   & X &   &   &  \\
    Objectives &   &   &   & X &   &   &  \\
    Abstract &   &   &   & X &   &   &  \\
    Scope and delimitation &   &   &   & X &   &   &  \\
    Review of related literature &   &   &   & X &   &   &  \\
    Methodology &   &   &   &   & X & X &  \\
    Individual Research &   &   &   &   & X & X & X \\
    Schematic diagram &   &   &   &   & X & X & X \\
    Balancing methods &   &   &   &   &   &   &  \\
    Sensors &   &   &   &   &   &   & X \\
    Sensors Calibration &   &   &   &   &   &   &  \\
    PIC programming &   &   &   &   &   &   & X \\
    Android programming &   &   &   &   &   &   & X \\
    Android layout &   &   &   &   &   &   &  \\
    OpenCV Integration &   &   &   &   &   &   & X \\
    Board design &   &   &   &   &   &   &  \\
    Board layout &   &   &   &   &   &   &  \\
    Fabrication &   &   &   &   &   &   &  \\
    Mounting &   &   &   &   &   &   &  \\
    Proofreading and Revisions &   &   &   &   &   &   &  \\
    Final documentation &   &   &   &   &   &   &  \\
    Defense &   &   &   &   &   &   &  \\
    \bottomrule
    \end{tabular}%
  \label{tab:addlabel}%
\end{table}%




% Table generated by Excel2LaTeX from sheet 'Sheet1'
\begin{table}[htbp]
  \centering
  \caption{Gannt Chart Part 2}
    \begin{tabular}{rrrrrrrrr}
    \toprule
      &       W7 &       W8 &       W9 &       W10 &       W11 &       W12 &       W13 &       W14 \\
    \midrule
    Research for a topic &   &   &   &   &   &   &   &  \\
    Submission of proposed topic &   &   &   &   &   &   &   &  \\
    Background of the study &   &   &   &   &   &   &   &  \\
    Statement of the problem &   &   &   &   &   &   &   &  \\
    Objectives &   &   &   &   &   &   &   &  \\
    Abstract &   &   &   &   &   &   &   &  \\
    Scope and delimitation &   &   &   &   &   &   &   &  \\
    Review of related literature &   &   &   &   &   &   &   &  \\
    Methodology &   &   &   &   &   &   &   &  \\
    Individual Research & X & X & X & X & X & X &   &  \\
    Schematic diagram &   &   &   &   &   &   &   &  \\
    Balancing methods &   &   &   &   &   &   &   &  \\
    Sensors & X & X &   &   &   &   &   &  \\
    Sensors Calibration &   &   & X & X & X & X &   &  \\
    PIC programming & X & X & X & X & X & X &   &  \\
    Android programming & X & X & X & X & X & X &   &  \\
    Android layout &   &   &   & X & X & X &   &  \\
    OpenCV Integration & X & X & X &   &   &   &   &  \\
    Board design &   &   &   &   &   &   &   &  \\
    Board layout &   &   &   &   &   &   &   &  \\
    Fabrication &   &   &   &   &   &   &   &  \\
    Mounting &   &   &   &   &   &   &   &  \\
    Proofreading and Revisions &   &   &   &   &   &   & X & X \\
    Final documentation &   &   &   &   &   &   & X & X \\
    Defense &   &   &   &   &   &   &   & X \\
    \bottomrule
    \end{tabular}%
  \label{tab:addlabel}%
\end{table}%

% Table generated by Excel2LaTeX from sheet 'Sheet1'
\begin{table}[htbp]
  \centering
  \caption{Estimated Budget}
    \begin{tabular}{rr}
    \toprule
    Laptop & 30000 \\
    \midrule
    Android Phone & 6000 \\
    Microcontroller & 250 \\
    Temperature Sensor & 85 \\
    Humidity Sensor & 400 \\
    PM2.5 Sensor & 1600 \\
    Carbon Monoxide Sensor & 350 \\
    TOTAL COST & 38685 \\
    \bottomrule
    \end{tabular}%
  \label{tab:addlabel}%
\end{table}%

\ifFinished
\else


\section{Publication Plan}
\blindtext

\fi


\section{Overview}

In the first chapter, it will be helpful for readers to understand what is the purpose of making the device and android application and why it is important for the society. It also shows how the project will be implemented in the real world from the hypothesis. For the second part of the paper, there will be a lot of helpful literature related to the air quality, discomfort index, respiratory health, prevention of lung cancer, effect of dust to the human body, circuit design for humidity, dust, and temperature sensors. These literature will guide the group what is the right way to develop a project and make it functional in order to fulfill the standard of the public. Theoretical considerations will be the key part to determine the data gathered from the device because there are theoretical standards in other research to know what are the air quality and discomfort index. Considering the design, it will be fully electronic design because the implementation in the hardware will be using electronic circuits. methodology will introduce how the data is gathered in the device and represented to the users. result and discussion will be providing the user feedback and the actual data given by the device in real situation. The value of this project will be determined in the conclusion based on all the provided data and actual simulation. It is the most important part to prove how this project fulfilled its purpose for the public health awareness.


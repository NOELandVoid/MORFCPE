\section{Concluding Remarks}

The common function of the Arduino base temperature and humidity meter is to detect the temperature and humidity in any places in order to come up with discomfort index value.  This discomfort computation feature of the Arduino base machine would be a new way of giving information on how students feel inside the university campus.
In this thesis, the group discussed the process and algorithms to detect temperature and humidity and come up with discomfort index value.  The problem that was faced during the development of this research was how to gather the ground truth data and compare it with the data gathered in order to know that it is accurate.  The first method used is analog and digital temperature and humidity meter data gathering.  In comparison, analog and digital humidity and temperature meters showed similar results but digital showed a more accurate data which is nearer to the data gathered in the Arduino temperature and humidity detector machine.  The ground truth was set as data from digital temperature and humidity meter to come up with a more accurate discomfort index result.  On the other hand, the android based application for discomfort index indication used firebase technology for displaying and uploading the humidity, temperature, and discomfort index data.  The data is uploaded in the firebase and displays the data from various locations.  The advantage of this application is that the crowd can easily upload their humidity and temperature data so that everyone application users will know which places are comfortable and which places are not. When this data is transferred into a google map, it will be a heat map of the campus to graphically indicate the discomfort index of different locations.
The project came up with several problems for the temperature and humidity meter machine. There was a challenge that how this machine will get accurate data from the sensor before the ground truth was set as reference. Spreading the informative data to the student was also a problem but the firebase crowd sourcing technology solved the issue. The Android platform application was a better option for the discomfort indicator due to its versatility and expandability, compared to other mobile development platforms.
The hardware presented in this thesis can be further developed into smaller size and come up with more sensors. It can be innovated with the use of dust sensor and carbon monoxide sensors to perform such functionality. This Arduino temperature and humidity meter machine can be of use not only in the campus, but also in any places outside of the university. This application can further touch the area of health awareness and medical information regarding the discomfort index and data gathered.


\section{Contributions}

The interrelated \index{contributions} contributions and supplements that have been developed in this \documentType  are listed as follows.

\begin{itemize}
  \item The construction of an accurate device that measures temperature and humidity
	
	\item The development of an Android application to increase social awareness
  
	
\end{itemize}


\section{Recommendations}

There is more to air pollution than measuring particulate matter and carbon monoxide. It is highly recommended that the measurements of air pollutants be improved by the addition of more sensors to the Arduino system so that more air pollutants and parameters can be measured such as sulfur dioxide and nitrogen dioxide. The system's setting so far is within the campus and the values shown are nearly consistent with one another. It is also recommended to further expand the coverage of taking down the discomfort index in order for more areas to be involved. Since Google Map API was used to take note of the location, another recommended study is to make use of the GPS location to mark that certain area's discomfort levels.

\section{Future Prospects}

There are several prospect related in this research that may be extended for further studies. So the suggested topics are listed in the following.

\begin{enumerate}
	\item  The addition of more air pollutants to be measured.
	
	\item  The expansion of areas that take note of temperature, humidity, and amount of air pollution.
		
\end{enumerate}



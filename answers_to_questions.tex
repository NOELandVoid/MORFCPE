


\section{How important is the problem to practice?}

The Philippines is a country that is prone to discomfort due to the inevitable elements of air pollution and rising heat levels. An Android application for awareness can be able to alert the locals about these issues.	
	
\section{How will you know if the solution/s that you will achieve would be better than existing ones?}	

Currently, there are no Android applications that provide real-time updates on discomfort index and amount of dust and carbon monoxide in a Philippine implementation.

\subsection{How will you measure the improvement/s?}	

Improvements could be measured by providing different ground truths (other thermometers/hygrometers) to test the accuracy of the system as well as surveys to confirm the level of discomfort felt by the user. Also, integrating the system to the phone is a way to retrieve data easily and it would occupy less space instead of having two separate systems communicating.
	
\subsubsection{What is/are your basis/bases for the improvement/s?}

The accuracy of the system will be the basis of improvement as well as the apparent level of discomfort felt by the user.
		
\subsubsection{Why did you choose that/those basis/bases?}

These data would not only test the accuracy of the Arduino system but also validate the data with the user's perceived level of discomfort.
				
\subsubsection{How significant are your measure/s of the improvement/s?}

They are significant because the measures of improvement will be more expensive than our system and will determine if a low cost system can be viable alternative to the existing systems.


\section{What is the difference of the solution/s from existing ones?}
	
Weather reports provide temperature and humidity in different parts of the world but our solution combines them both into a discomfort index derived from heat which is an essential factor in the levels of comfort of an individual.

\subsection{How is it different from previous and existing ones?}

Weather stations provide measurements pertaining to temperature and humidity but in this solution, the measurements can be accurately measured with an Arduino-based system. The crowd sourcing element in the research enables these data to be updated time and time again, faster than an selecting an interval of a daily update.

	
\section{What are the assumptions made (that are behind for your proposed solution to work)?}
	
For this research, it is assumed that almost every person in the community owns an Android phone because with this phone, one can access the information from the firebase database.
		
	
\subsection{Will your proposed solution/s be sensitive to these assumptions?}
	
Yes. The entire system designed so far is made for Android phones that are able to access this firebase database.

  
\subsection{Can your proposed solution/s be applied to more general cases when some of the assumptions are eliminated? If so, how?}

In the case of this study, the proposed solution cannot be applied to more general cases. The main backbone of the thesis is the Android system since it gathers the data from the Arduino system and it enables access to the different discomfort indices within the university.


\section{What is the necessity of your approach/proposed solution/s?}

Our solution aims towards the convenience of anyone that has the
	
\subsection{What will be the limits of applicability of your proposed~solution/s?}

As of now, the whole crowdsourcing system is implemented to provide data such as temperature and humidity for various locations within the university only.
					  
\subsection{What will be the message of the proposed solution to technical people?  How about to non-technical managers and business men?}
			
For the technical people, the message would be that it is possible to create an application that uses crowdsourcing to take note of air pollution and discomfort index by the construction of an Arduino system that can transmit data via Bluetooth to an Android device which can pass on the data to the firebase database accessible by anyone who has the application. For the non-technical managers, we would say that an application that takes note of real-time updates of the amount of discomfort based on heat and air pollution has been developed.



\section{How will you know if your proposed solution/s is/are correct?}

The sensors for temperature, humidity, amount of particulate matter, and carbon monoxide content will be tested based on the accuracy in terms of a ground truth. All group members that own Android phones can be able to verify the data.
			
\subsection{Will your results warrant the level of mathematics used (i.e., will the end justify the means)?}
	    
Yes. A mathematical formula in computing the discomfort index that makes use of temperature and humidity was used.
			

\section{Is/are there an/\_ alternative way/s to get to the same solution/s?}

Other microcontroller systems can be considered as alternatives since they also can be able to retrieve values of temperature and humidity with the sensors and transmit the data via Bluetooth or even another method of data transmission.
	
\subsection{Can you come up with illustrating examples, or even better, counter examples to your proposed solution/s?}

In terms of data gathering, the data would vary based on the time the measurements were taken and weather conditions. There are different stations and air quality devices present today such as Netatmo and CubeSensor however these are very expensive to implement.
	
\subsection{Is there an approximation that can arrive at the essentially the same proposed solution/s more easily?}
	
Integrating the system to the smartphone is a way to retrieve data easily and it would occupy less space instead of having two separate systems communicating.
	
	
\section{If you were the examiner of your proposal, how would you present the proposal in another way?}
	
It seems that it would be better if there would be a live system and app demonstration instead of the usual Powerpoint presentation in order to better understand how the system works.
	
\subsection{What are the weaknesses of your proposal?}

The system implemented within the university would yield nearly the same results from different locations.
